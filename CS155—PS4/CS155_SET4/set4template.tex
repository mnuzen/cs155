\newif\ifshowsolutions
\showsolutionstrue
\input{preamble.tex}



%%%%%%%%%%%%%%%%%%%%%%%%%%%%%%
% HEADER
%%%%%%%%%%%%%%%%%%%%%%%%%%%%%%

\chead{
  {\vbox{
      \vspace{2mm}
      \large
      Machine Learning \& Data Mining \hfill
      Caltech CS/CNS/EE 155 \hfill \\[1pt]
      Set 4\hfill
      January 2020. \\
    }
  }
}

\begin{document}
\pagestyle{fancy}



%%%%%%%%%%%%%%%%%%%%%%%%%%%%%%
% PROBLEM 1
%%%%%%%%%%%%%%%%%%%%%%%%%%%%%%

\newpage
\section{Deep Learning Principles [35 Points]}
\materials{lectures on deep learning}

\begin{problem}[5]
  Backpropagation and Weight Initialization Part 1
\end{problem}

\begin{subsolution}

\end{subsolution}

\newpage

\begin{problem}[5]
  Backpropagation and Weight Initialization Part 2
\end{problem}

\begin{subsolution}

\end{subsolution}

\newpage


\problem \textbf{[10 Points]}


\begin{solution}

\end{solution}

\newpage




\problem Approximating Functions Part 1 \textbf{[7 Points]}

\begin{subsolution}

\end{subsolution}

\newpage


\problem Approximating Functions Part 2 \textbf{[8 Points]}

\begin{subsolution}

\end{subsolution}


% problem 2
\newpage
\section{Depth vs Width on the MNIST Dataset  [25 Points]}

\problem \textbf{Installation} \textbf{[2 Points]}

\begin{solution}

torch:

torchvision:

\end{solution}

\newpage



\problem \textbf{The Data} \textbf{[3 Points]}

\begin{subsolution}

\end{subsolution}

\newpage



 \problem \textbf{Modeling Part 1} \textbf{[8 Points]}

\begin{solution}

\end{solution}

\newpage


 \problem \textbf{Modeling Part 2} \textbf{[6 Points]}

 \begin{solution}

\end{solution}

\newpage


  \problem \textbf{Modeling Part 3} \textbf{[6 Points]}

  \begin{solution}

\end{solution}

 \newpage
 % problem 3
 \section{Convolutional Neural Networks  [40 Points]}
 \problem Zero Padding \textbf{[5 Points]}

\begin{solution}

\end{solution}

\newpage


\subsection{5 x 5 Convolutions}


\problem[2]

\begin{subsolution}

\end{subsolution}

\newpage


\problem[3]

\begin{subsolution}

\end{subsolution}

\newpage


 \subsection{Max/Average Pooling}

\problem[3]

\begin{subsolution}

\end{subsolution}

\newpage


\problem[3]

\begin{subsolution}

\end{subsolution}

\newpage


\problem[4]

\begin{subsolution}

\end{subsolution}

\newpage


\subsection{PyTorch implementation}
\problem[20]


\begin{subsolution}

\end{subsolution}

\end{document}
